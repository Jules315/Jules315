\documentclass[12pt]{article}
\usepackage{tikz}
\usetikzlibrary{matrix,arrows}
\usepackage{graphicx}
\usepackage{amsmath} 
\usepackage{amssymb} 
\usepackage{amsthm}
\usepackage{enumitem} % to change appearance of enum and item environments
\usepackage{framed}
\usepackage{color}
\usepackage{multicol}
\usepackage{minted}

%\usepackage{yhmath}
% adjust the margins using the geometry package
\usepackage[left=0.75in, right=0.75in, top=0.75in, bottom=0.75in]{geometry}
\usepackage[parfill]{parskip}
 %\usepackage{mathpazo}
%\usepackage{euler}

% customize the headers using the fancyhdr package
\usepackage{fancyhdr}
\pagestyle{fancy}

\usepackage{hyperref}

\usepackage{mathrsfs}
\usepackage{mathtools}

\renewcommand{\headrulewidth}{0.4pt}
\renewcommand{\footrulewidth}{0.4pt}

\newenvironment{hwproblem}[1]{{\large \bfseries Problem #1\,:} \begin{trivlist}\item[]\vspace{-0.5ex}}{\end{trivlist}\vspace{3ex}}
\newenvironment{hwproblemquestion}[1]{{\bfseries Question #1\,:} \begin{trivlist}\item[]\vspace{-1.5ex}}{\end{trivlist}\vspace{3ex}}
\newenvironment{lecexercise}[1]{{\large \bfseries Exercise #1\,:} \begin{trivlist}\item[]\vspace{-0.5ex}}{\end{trivlist}\vspace{3ex}}
\newenvironment{note}[1]{{\large \bfseries Note #1\,:} \begin{trivlist}\item[]\vspace{-0.5ex}}{\end{trivlist}\vspace{3ex}}
\newenvironment{hwsubpart}[1]{{\bfseries #1\,:} \begin{trivlist}\item[]\vspace{-0.5ex}}{\end{trivlist}\vspace{3ex}}


% create theorem style
\newtheoremstyle{break}% name
  {}%         Space above, empty = `usual value'
  {}%         Space below
  {}%         Body font
  {}%         Indent amount (empty = no indent, \parindent = para indent)
  {\bfseries}% Thm head font
  {.}%        Punctuation after thm head
  {\newline}% Space after thm head: \newline = linebreak
  {}%         Thm head spec

\theoremstyle{break}
\newtheorem{hwquestion}{Question}

\newtheorem{theorem}{Theorem}
\numberwithin{theorem}{subsection}

\newtheorem*{lemma*}{Lemma}
\newtheorem{lemma}{Lemma}
\numberwithin{lemma}{subsection}

\newtheorem{corollary}{Corollary}
\numberwithin{corollary}{subsection}


\newtheorem*{definition}{Definition}

\newtheorem*{note*}{Note}

\newtheorem*{remark}{Remark}

\newtheorem*{hint}{Hint}

\newtheorem*{example}{Example}

\numberwithin{equation}{subsection}

%\addbibresource{ref.bib} %Imports bibliography file

% customize enumerate and itemize environments
\setlist[itemize]{labelsep=1ex,itemsep=1.5ex,parsep=0ex,leftmargin=4ex,topsep=0.5ex}
\setlist[enumerate]{labelsep=1ex,itemsep=1.5ex,parsep=0ex,leftmargin=4ex,topsep=0.5ex}


\rhead{Behavioral Economics}
\lhead{Papers} 



\lfoot{}
\cfoot{} 
\rfoot{\thepage}

\setlength{\headheight}{15pt}
\title{Papers} 
\date{}
\author{}

\begin{document}

\section{Behavioral game theory}

\subsection{Dufwenberg (2002): Marital investments, time consistency and emotions}

\subsubsection{general}
\begin{itemize}
    \item Introduces 'psychological game theory' where parameters enter the utility functions that try to model the (dis)utility of emotional responses.
    \item In Duwenberg's marital investment game proclivity to guilt, denoted by $\gamma \geq 0$, is such a parameter. The effect of the parameter depends on beliefs.
    \item the probability the husband stays is given by $\tau \in [0,1]$, the first order belief of the wife that the husband stays is given by $\tau^{\prime} \in [0,1]$ (wife's trust), the second order belief of the Husband (husband's expectation of the wife's trust) is given by $\tau^{\prime\prime} \in [0,1]$.
    \item Sub-game perfection leads to the inefficient outcome where the Wife chooses not to marry (shown in game without guilt, where also nature first decides whether its a good or bad match). However empirical evidence on trust games shows that often efficient outcomes can be sustained.
    \item Dufwenberg shows that the efficient equilibrium can be sustained when the husband has a  sufficiently high proclivity to guilt.
\end{itemize}

\begin{figure}[H]
    \centering
    \includegraphics[width=\textwidth/3]{dufwenberg_game.png}
    \caption{Dufwenberg's trust game}
    \label{combine}
\end{figure}

\end{document}